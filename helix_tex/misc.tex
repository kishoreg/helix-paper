\section{Miscellaneous}
Here is stuff that may be needed later.


{\em Take an example here to map the terms.  Most distributed systems
at a high level can be tasks, partitions/replicas and its state. In
this paper, we use these terminolgies to explain Helix design and
implementation. Its important to note that these are fairly generic
terms and can be mapped to different things based on the system. For
example, in a data serving system each task can be a database and if
the database is sharded, each partition is a shard. Similarly in a pub
sub system, a task can be a Topic/Queue and in a search system it can
refer to an Index.  }

%------------------------------------------------
Helix needed a way to map partitions to nodes.
tasks, partition/replica and its state.
Given that partitions are distributed among various nodes in the
cluster for performance, scalability and operability, the first {\em
  thing} that Helix needed was a various partition assignment algorithms. 
One of the commonly used
algorithm is consistent hashing. The most important goal on any
assignment is distribute the partitions evenly(assuming they are equal
weight) among all the nodes in the cluster.




In this section we will cover how the state
machine with constraints is
satisfied in Helix in a generic way. As we see
in our previous section one would
simply provide the following for each resource
in the system
* Number of partitions
* Number of Replicas for each partition
* State Machine with valid states and state
* transitions
* Constraints on state and state transition.

{\em 
Justify any interesting design decisions, and the theme is to
emphasize how they help Helix be generic.

Think about giving anecdote about how we easily spun Espresso w/mysql
replication from Espresso w/databus replication.  Repeat this in
evalation.

Limitations/negatives
\bi
\item What are the possible problems?
\item Why aren't we worried about them right now?
\item What are possible solutions if we want to address them.
\ei


Note about ideal state terminology.
\bi
\item 
We want to describe the set of constraints and
optimization goals as 1 thing, and an actual
config that achieves those goals as a 2nd
thing.
\item The first thing can be described as an
optimization problem: maybe "optimization
spec?"
\item The second thing is a solution: maybe
``assignment''
\ei


Consider merged impl with this section if it makes sense


Even though all distributed systems share the common terminologies,
they differ from each other in the way partitions are distributed
among nodes, states associated with each replica of the partition.
}

\section{Formalisms}

This section formalizes the description of a DDS, informally introduced
in Section~\ref{Informal}.

\begin{definition}
Let \(R = \{r_1, r_2, \ldots r_{N_R}\}\) be a set of resources. 
\end{definition}

\begin{definition}
Let the number of resources be \(N_R\)
\end{definition}

\begin{definition}
Each resource \(r \in R\) has a set of partitions
\(P(r) = \{p^r_1, p^r_2, \ldots p^r_{N_(r)}\}\).
\end{definition}

\begin{definition}
Let \(N(r)\) be the number of partitions for resource \(r \in R\). 
\end{definition}

\begin{definition}
Let \(M = \{m_1, m_2, \ldots m_{N_I}\}\) be a set of machines or 
or nodes or instances.  
\end{definition}

\begin{definition}
Let \(N_M\) be the number of machines.
\end{definition}

We group the nodes into a mutually exclusive and exhaustive set of zones.

\begin{definition}
Let \(Z = \{z_1, z_2, \ldots z_{N_Z}\}\) be a set of  zones such that 
(i) \(z_i \subseteq M\) 
(ii) \(\forall i \forall j , i \neq j \Rightarrow z_i \cap z_j = \phi \) 
(iii) \(\cup_i z_i = M\)
\end{definition}

\begin{definition}
Let \(N_Z\) be the number of zones.
\end{definition}

Since resources use machines in common, there is clearly interplay
between actions performed on different resources. Nonetheless, at this
point in the discussion, we consider resources independently. 
We focus on a single resource and write the set of \(N_P\) partitions 
as \(P = \{p_1, p_2, \ldots p_{N_P}\}\). 

\begin{definition}
\label{defn_states}
Let \(S\) be a set of states.  
\end{definition}

We define the state of the system when it is quiescent as follows.

\begin{definition}
\label{defn_q1_system_state}
The {\bf state} \(U\) of a system at a given point in time
is defined as a set of tuples, 
\(\{(p, s, m)\} \).  A tuple \((p, s, m)\) tells us that
partition \(p \in P\) exists in state \(s \in S\) on machine \(m \in M\). 
\end{definition}

%-----------------------------------------------------
We {\em must} define which state transitions are legal 
--- Definition~\ref{defn_legal_transition}.
\begin{definition}
\label{defn_legal_transition}.
Let \(S_L \subseteq S^2 =  S \times S\) be a set of
legal state transitions. For example, \((s_i, s_j) \in S_L
\Rightarrow\) it is legal for a partition to transition from state 
\(s_i\) to state \(s_j\).
\end{definition}
%-----------------------------------------------------
We {\em may} define a partial ordering on the state transitions.
\begin{definition}
\label{defn_state_trans_partial_order}
Let \(S_P \subseteq S_L \times S_L\) be a set of state transitions such
that \(((s_1, s_2), (s_3, s_4)) \in S_P\) indicates that the transition
\((s_1, s_2)\) should be executed in preference to the transition
\((s_3, s_3)\).
\end{definition}
%-----------------------------------------------------

We {\it may} impose requirements for each \(s
\in S\) as in Invariant~\ref{inv_state_count_bound1}
\begin{definition}
Let \(Count(p', s') = |\{(p, s, m) | (p, s, m) \in U \wedge p = p' \wedge s =
s'\}|\) be the number of machines in which partition \(p'\) is in state
\(s'\)
\end{definition}


\begin{invariant}
\label{inv_state_count_bound1}
\(R_X(s) = (lb, ub) \Rightarrow \forall p, lb \leq Count(s, p) \leq ub
\) says that the number of machines in state \(s\) is {\bf expected} to
be between \(lb\) and \(ub\) on all partitions.
\end{invariant}

\begin{invariant}
\label{inv_one_state_for_one_partition_on_one_machine}
Note that a partition cannot be in 2 different states on the same
machine i.e., \((p_1, m_1, s_1) \in U \Rightarrow (p_1, m_1, s_2) \not
\in U\).
\end{invariant}

Since the behavior of
one partition can be reasoned about independent of the state of other
partitions, it simplifies notation to assume that there is only one
partition. We can 
re-write Definition~\ref{defn_q1_system_state} as 
Definition~\ref{defn_q_system_state}, 

\begin{definition}
\label{defn_q_system_state}
The {\bf state} \(U\) of a system at a given point in time
is defined as a set of tuples, 
\(\{(m, s)\} \).  A tuple \((m, s)\) tells us that
machine \(m \in M\).  is in state \(s \in S\) on 
\end{definition}

When Helix is running, there are external events that cause the state of
the system to be undesirable. For example, a machine may go down or
become over-loaded, another machine may come on-line or may become
lightly loaded. It is the job of Helix to respond to external stimuli by
changing the state of the system. In doing so, it is important to
remember what Brutus\footnote{Julius Caesar, Act II, Scene i},  
eloquently stated as 
\begin{verse}
{\em 
Between the acting of a dreadful thing \\
And the first motion, all the interim is \\ 
Like a phantasma or a hideous dream; \\
}
\end{verse}

What does this mean in our context? It means that when one issues a
directive to a machine to change state, there is an unpredictable amount
of time that elapses between the issuance of the directive and the
notification of it having taken effect. The kind of change we can
request is in Definition~\ref{defn_change}.

\begin{definition}
\label{defn_change}
A change is a tuple of the form \((m, s_1, s_2)\) indicating a (legal) request to
change machine \(m\) from state \(s_1\) to state \(s_2\). Note that
\(s_1 \neq s_2\) and and \((s_1, s_2) \in S_L\)
\end{definition}

\begin{definition}
\label{defn_apply_change}.
The application of a change \((m, s_1, s_2)\) to a quiescent state \(U\) 
results (at some future point in time) in a new state \(U'\) such that 
\((m, s_2) \in U'\), assuming that \( (m, s_1) \in U\).
\end{definition}

\begin{definition}
\label{defn_system_change}
Let the ``change set'' \(U_D\) be the set of change requests (Definition~\ref{defn_change}). 
that have been issued but have not yet been processed. We can also view
this as the dynamic analog of Definition~\ref{defn_q_system_state}. Note
that there can be {\bf only one} change request for a given machine 
\end{definition}

Given that there is uncertainty in the state of a machine, we
re-write Invariant~\ref{inv_state_count_bound1} as 
Definition~\ref{defn_d_state_count}.

\begin{definition}
\label{defn_d_state_count}
\(R_D(s) = (s, lb, ub) \Rightarrow lb \leq Count(s) \leq ub
\) says that the number of machines in state \(s\) in a {\bf D}ynamic
system {\bf is} between \(lb\) and \(ub\).
\end{definition}

The overall state count is \(R(D)\), defined in
Definition~\ref{defn_d_state_count_overall}
\begin{definition}
\label{defn_d_state_count_overall}
\(R_D = \{R_D(s)\}\)
\end{definition}

So, we model the flux that Helix must contend with as a dance between 
\bi
\item Helix issuing change requests (Definition~\ref{defn_change}) that get
added to \(U_D\) (Definition~\ref{defn_system_change}). This 
increases the uncertainty in our knowledge of the system. 
In terms of our notations, the
difference between \(lb\) and \(ub\) for some \((s, lb, ub) \in R_D\)
increases.
\item the system responding with notifications that decreases the
uncertainty.
In terms of our notations, the
difference between \(lb\) and \(ub\) for some \((s, lb, ub) \in R_D\)
decreases.
\ei

We can model this as a producer/consumer relationship where Helix is the
producer and the system is the consumer and the result of a change is
defined in Definition~\ref{defn_apply_change}. Consider a change request
of the form \((m, s_1, s_2)\) 
\bd
\item [PRODUCER] Change request is made
\be
\item \((m, s_1, s_2)\) is added to \(U_D\). 
\item Change to \(R_D\) is as follows
\bd
\item [PRIOR] \((s_1, l_1, u_1), (s_2, l_2, u_2) \in R_D\) 
\item [POSTERIOR] \((s_1, l_1-1, u_1), (s_2, l_2, u_2+1) \in R_D\)
\ed
\ee
\item [CONSUMER] Change request is satisifed
\be
\item \((m, s_1, s_2)\) is deleted from \(U_D\) and 
\bd
\item [PRIOR] \((s_1, l_1-1, u_1), (s_2, l_2, u_2+1) \in R_D\)
\item [POSTERIOR] \((s_1, l_1-1, u_1-1), (s_2, l_2+1, u_2+1) \in R_D\)
\ed
\ee
\ed

\subsection{Executing a change set}

\subsubsection{Motivation}
We have decribed what happens when a single change is applied. Assume
that we had a set of changes to apply. 
We would like to execute as many of these changes in parallel. However,
when we execute them in parallel, we have no guarantee on the order in
which they will execute. Which means that unless we are confident that
{\em all} execution orders are valid, we should {\em not} run them in
parallel.  

The user {\it may} also impose an ordering on state transitions. In
other words, if have a choice between executing change \((m_1, s_{11},
s_{12})\) and \((m_2, s_{21}, s_{22})\), then we execute the
change with the higher priority --- Definition~\ref{defn_priority}

\begin{definition}
\label{defn_priority}
There exists a many-to-one mapping from \(P: S_L \rightarrow I^+\) such
that \\
\(((s_{11}, s_{12}), p_1), (s_{21}, s_{22}), p_2)) \in P
\wedge p_1 < p_2 \Rightarrow (s_{11}, s_{12}) \) has higher priority
than \((s_{21}, s_{22}) \) 
\end{definition}

We are now in a position to state the problem of executing a change set
in parallel.
\begin{definition}
\label{defn_the_problem}
Given the state of the system, in terms of \(R_D\) and \(U_D\) and 
a change set, find the largest subset of it which respects
Definition~\ref{defn_priority} and
Invariant~\ref{inv_state_count_bound1}.
\end{definition}

%% On a more pragmatic note, we must bear in mind that we are working
%% against a dynamic system where machines can go down and come back up in
%% unpredictable ways. So, there is little point in spending a lot of time
%% planning a perfect execution plan and then having it come to naught
%% because of external factors.
%% This was stated more eloquently as \footnote{work of devotion
%% written in Latin by Thomas Kempis.}
%% \begin{center}
%% {\em 
%% Man proposes but God disposes
%% }
%% \end{center}

\begin{definition}
A row \((s_1, s_2, n, r)\) of Table~\ref{State_Transition_Count}
indicates that there are \(n\) machines in \(U_D\) that need to
transition from state \(s_1\) to \(s_2\)
\end{definition}

Figure~\ref{algo} presents the algorithm for Definition~\ref{defn_the_problem}.

\begin{figure}
\centering
\fbox{
\begin{minipage}{8 cm}
\centering
\begin{tabbing} \hspace*{0.25in} \= \hspace*{0.25in} \= \kill
1. \> Create Data Structure in Table~\ref{State_Transition_Count}, sorted
descending on priority \\
\> Process the table a row at a time. Let \(m'\) be the number of
machines \\ 
2. \> For each row, find the largest value of \(m \leq m'\) such that \\
\> \(m\) can be applied without violation of any constraint
\end{tabbing}
\end{minipage}
}
\label{algo}
\caption{Algorithm A for Definition~\ref{defn_the_problem}}
\end{figure}

\begin{table}
\centering
\begin{tabular}{|l|l|l|l|} \hline \hline
{\bf From State} & {\bf To State} & {\bf Number of} & {\bf Priority} \\ 
                 &                & {\bf Machines} &                \\ \hline \hline
\(s_1\) & \(s_2\) & 3 & 1 \\ \hline 
\(s_2\) & \(s_3\) & 4 & 2 \\ \hline 
\(s_1\) & \(s_3\) & 1 & 2 \\ \hline 
--- & --- & --- & --- \\ \hline 
\hline
\end{tabular}
\caption{Sample of Count of Transitions to be made}
\label{State_Transition_Count}
\end{table}
%------------------------------------------------------
\begin{theorem}
The complexity of algorithm A is \(O(M_P)\). This follows from the fact 
that the maximum number of state transitions that can occur for that
partition.
\end{theorem}

\begin{proof}
\TBC
\end{proof}

\section{Modeling Usage Examples in Helix}

\subsection{\GP\ in Helix}
\label{Galapagos_in_Helix}
When deployed as part of \GP (Section~\ref{Galapagos}), Helix is
configured as 
\be
\item \(S = \{M, S, O\}\), where \(M =\) Master, \(S =\) Slave, \(O =\)
Offline (Definition~\ref{defn_states}) {\bf TO BE CORRECTED}
\item When a node comes online, it generates significant network
traffic. Throttling network traffic is modeled as \(R_D(S) = (S, lb, ub)
\Rightarrow ub - lb \leq 3\) (Definition~\ref{defn_d_state_count})
\item \TBC
\ee

\subsection{YARN in Helix}
\label{YARN_in_Helix}

Describe how Helix {\it could} have been used by YARN \TBCKG
%------------------------------------------------
\section{Operational Ease}
\label{Operational_Ease}

Having defined how our system {\it should} operate, we would like to
know how the system {\it does} operate.  For example,
Section~\ref{Espresso_in_Helix} indicates that there should be exactly
one master for each partition. From an Ops perspetive though (knowing
that nodes fail), it is more important to know {\em how often} this
was violated than {\em whether} it was violated.

In general, we would like to know the fraction (percentage) of time
during which there were \(n\) instances in state \(S_1\) for \(n =
0,1, \ldots\). Since this is different for different partitions, we
sum the times across the partitions.
Table~\ref{tbl_master_time_stats} shows us that the percentage of time
when a partition had no master was 2.69\%. Whether this is acceptable
or not depends on the SLA that the Ops team has established with the
business owner. What is important to note is that
\be
\item This kind of log analysis comes ``out-of-the-box'' with Helix
\item Helix makes it easy to have such a structured discussion.
\ee
\begin{table}[h]
\centering
\begin{tabular}{|l|l|} \hline \hline
{\bf Number of Instances} & {\bf Time (in sec) } \\ \hline \hline
0 & 20384093 \\ \hline 
1 & 734689509 \\ \hline 
\hline 
\end{tabular}
\caption{Time Statistics for state MASTER}
\label{tbl_master_time_stats}
\end{table}

% \subsection{Time Statistics for State Slave in \ES}
% \label{time_stats_for_slave}
% See Table~\ref{tbl_slave_time_stats}
% \begin{table}[h]
% \centering
% \begin{tabular}{|l|l|} \hline \hline
% {\bf Number of Instances} & {\bf Time (in sec) } \\ \hline \hline
% 0 & 9586805 \\ \hline 
% 1 & 23092725 \\ \hline 
% 2 & 721804536 \\ \hline 
% 3 & 589536 \\ \hline 
% \hline 
% \end{tabular}
% \caption{Time Statistics for state SLAVE}
% \label{tbl_slave_time_stats}
% \end{table}

%

