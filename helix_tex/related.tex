\section{Related Work}
\label{sec:related}
%
The problems that systems dealing with cluster management attempt to solve fall in 3 main categories:
(a) resource management, (b) enforcing correct system behavior in the presence
of faults and other changes, and (c) monitoring. 

There are generic systems that solve resource management and monitoring, but not
fault-tolerance, in a generic manner.  Other DDSs support fault-tolerance and
enforce system behavior correctness in a way that is specific to that particular
DDS.

\subsection{Generic Cluster Management Systems}

Systems such as YARN~\cite{yarn} and Mesos~\cite{mesos} implement resource management and 
some monitoring capabilities in a generic way. Applications using these systems are responsible 
for implementing the specific logic to react to changes in the cluster and enforce correct system behavior. 
We detail Mesos an example.

Mesos has 3 components: \emph{slave daemons} that run on each cluster node, a
\emph{master daemon} that manages the slave daemons, and \emph{applications} that 
run tasks on the slaves.
The master daemon is responsible for allocating resources (cpu, ram, disk) across
the different applications. It also provides a pluggable policy to allow for
adding of new allocation modules.
An application running on Mesos consists of a \emph{scheduler} and an
\emph{executor}.  The scheduler registers with the master daemon to request
resources. The executor process
runs on the slave nodes to run the application tasks. The master daemon determines how many
resources to offer to each application. The application scheduler is responsible
for deciding which offered resources to use.  Mesos then launches the tasks on
the corresponding slaves.
Unlike Helix, Mesos does not provide a declarative way for applications to
define their behavior and constraints on it and have it enforced automatically,
but provides resource allocation functionality that is complementary to Helix.

\subsection{Custom Cluster Management}

As mentioned before, distributed systems like PNUTS~\cite{cooper08},
Hbase~\cite{hbase}, HDFS~\cite{hadoop}, and MongoDB~\cite{mongodb} implement 
cluster management in those specific systems. As an example, we examine
MongoDB's cluster management. 

MongoDB provides the concept of document collections which are sharded into
multiple chunks and assigned to servers in a cluster. MongoDB uses range
partitioning and splits chunks when chunks grow too large. When the load of any
node gets too large, the cluster must be rebalanced by adding more nodes to the
cluster and some chunks move to the new nodes.

Like any other distributed system, MongoDB must provide failover capability and
ensure that a logical shard is always online. To do this, MongoDB assigns n
servers (typically 2-3) to a shard. These servers are part of a replica set.
Replica sets are a form of asynchronous master-slave replication and consist of
a primary node and slave nodes that replicate from the primary. When the primary
fails, one of the slaves is elected as the primary.  This approach to fault
tolerance is similar to the MasterSlave state model \ES uses; \ES gets it for
free from \helix, while MongoDB had to build it.

Other distributed systems implement very similar cluster management
capabilities, but each system re-implements it for itself, and not in a way that
can be easily reused by other systems. 

\subsection{Domain Specific Distributed Systems}

Hadoop/MapReduce~\cite{hadoop,dean04} provides a programming model and
system for 
processing large data sets. It has been wildly successful, largely because it
lets programmers focus on their jobs' logic, while masking the details of the
distributed setting on which their jobs run.  Each MR job consists of multiple map and reduce jobs 
and the processing slots get assigned to these jobs. Hadoop takes care of monitoring 
the tasks and restarting tasks in case of failures.  Hadoop, then, very
successfully solves the problems we outline for cluster management, but only in
the MR context.  \helix aims to bring this ease of programming to DDS
development.

Zookeeper~\cite{zookeeper} is a centralized service for maintaining configuration information, 
naming, providing distributed synchronization, and providing group services. 
It is used by many distributed systems to help implement cluster management
functionality, and we of course use it heavily as a building block in \helix.  It is important to
note that Zookeeper alone does not solve the cluster management problems.  For
example, it provides functionality to notify a cluster manager when a node has
died, but does not plan and execute a response. 

\subsection{Distributed resource management}
%
A number of older systems pioneered solutions in distributed resource
management.  Amoeba~\cite{tanenbaum90}, Globus~\cite{frey02}, GLUnix~\cite{ghormley98} and 
Legion~\cite{chapin99} all manage large numbers of
servers and present them as a single, shared resource to users.  The challenges
they tackle most relevant to \helix include placing resources, scheduling jobs, monitoring 
jobs for failures, and reliably disseminating messages among nodes.  Much of the
scheduling work relates closely to how we distribute resource partitions among
nodes, and how we hope to make this distribution more dynamic in the future,
where \helix will move partitions aggressively in response to load imbalance.
On the other hand, we address failure monitoring in \helix with Zookeeper,
which itself relates closely to these systems.  In general, it is easy to see
the lineage of these systems reflected in \helix.
