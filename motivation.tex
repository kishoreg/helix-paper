\section{Motivation}
\label{sec:motivation}
%
At \linkedin we have built and deployed a number of DDSs.
These DDSs share many common attributes in how they are deployed, monitored, expanded, etc.  
However, these systems have differences in
 specific optimization goals and constraints governing those common attributes.
The goal of building a general cluster manager is very appealing for \linkedin and the community.  
Providing a common way for a system to declare 
its own model, without worrying about how to force the system to conform to that model, makes creating 
that model easy and uniform.  Moreover, it simplifies operational support, since there is only one 
cluster manager executing cluster changes across different systems, and we can concentrate on that 
manager's correctness.

  Our goal in this work is to 
formally identify the common characteristics of DDSs and then build a cluster manager that 
both handles those requirements for the DDS, while exposing interfaces so the DDS can declaratively
describe their own system-specific model of behavior. The cluster manager then enforces the system-specific model of behavior. 

 

\subsection{Use Cases}
\label{sec:usecases}
%
Before delving into cluster manager requirements, it helps to review a sample of DDSs that we 
target.  We describe three \linkedin systems.  We have purposefully chosen them for their diversity,
to ensure the commonalities we draw from them are truly found in most or all
DDSs.  These use cases are good examples, and are also some of the earliest use
cases at \linkedin that motivated our cluster management work.


\subheader{Espresso}
Espresso is a distributed, timeline consistent, scalable, document store that 
supports local secondary indexing and local transactions.  Espresso runs on a 
number of \emph{storage node} servers that store and index data and 
answer queries. Espresso databases are horizontally partitioned into a number of partitions, 
with each partition having a certain number of replicas distributed across the 
storage nodes.

Espresso designates one replica of each partition as \emph{master} and the rest as \emph{slaves}; only
one master may exist for each partition at any time.  
Espresso enforces timeline consistency where only the master of a
partition can accept writes to its records, and all slaves receive and apply the same writes through a replication stream.  For load balancing, both master and slave
partitions are assigned evenly across all storage nodes.  For fault tolerance, it adds the constraint that no   
two replicas of the same partition may be located on the same node.

To maintain high availability, Espresso must ensure every partition has an assigned master in the cluster. 
When a storage node fails, all partitions mastered on that node must have their mastership transferred to other
nodes and, in particular, to nodes hosting slaves of those same partitions. 
When this happens, the load should be as evenly distributed as possible.

Espresso is elastic; we add storage nodes as the number/size of databases
or the request rate against them, increases.
When nodes are added we migrate partitions from existing nodes to new ones.
The migration must maintain balanced distribution in the cluster but also minimize unnecessary movements. 
Rebalancing partitions in Espresso requires copying and transferring significant
amounts of data, and minimizing this expense is crucial.  Even when minimized, we must throttle rebalancing so 
existing nodes are not overwhelmed.

Skewed workloads might cause a partition, and the storage
node hosting it, to become a hot spot.  Espresso must execute partition moves to
alleviate any hot spots.

\subheader{\seas}
\linkedin's \seas lets internal customers define custom indexes on a
chosen dataset and then makes those indexes searchable via a service API.
The index service runs on a cluster of machines.  The index is broken into partitions
and each partition has a configured number of replicas.  Each cluster server runs 
an instance of the \sensei~\cite{sensei} system (an online index store) 
and hosts index partitions. Each new indexing service gets assigned
to a set of servers, and the partition replicas must be evenly distributed
across those servers. 

Unlike Espresso, search indexes are not directly written by external
applications, but instead subscribe to external data streams such as 
\databus and Kafka~\cite{linkedin12} and take their writes from them.  Therefore, 
there are no master-slave dynamics among the replicas of a partition; all replicas
are simply \emph{offline} or \emph{online}.  

When indexes are bootstrapped, the search service uses snapshots of the data
source (rather than the streams) to create new index partitions.   Bootstrapping 
is expensive and the system limits the number of partitions that may
concurrently bootstrap.  
\eat{%%%%%%
\seas has a few additional requirements from cluster management.  These include
peer-to-peer messaging capability (\aes{if we don't discuss this more later,
omit here}), support for instance level configuration (\aes{what does this
mean?}), and monitoring at node,partition and service granularities.
}%%%%%%%%

\subheader{\databus}
\databus is a \emph{change data capture} (CDC) system that provides a common pipeline 
for transporting events from LinkedIn primary databases to caches within various applications. 

\databus deploys a cluster of relays that pull the change log from multiple databases and let consumers subscribe to the 
change log stream.  Each \databus relay connects to one or more database servers
and hosts a certain subset of databases (and partitions) from those database servers. 
\databus has the same concerns as \ES and \seas for assigning databases and
partitions to relays.

\databus consumers have a cluster management problem as well.  For a
large partitioned database (e.g. \ES), the change log is consumed by a bank of
consumers.  Each databus partition is assigned to a consumer such that
partitions are evenly distributed across consumers and each partition is
assigned to exactly one consumer at a time. The set of consumers may grow over
time, and consumers may leave the group due to planned or unplanned outages.  In 
these cases, partitions must be reassigned, while maintaining balance and the
single consumer-per-partition invariant. 

\subsection{Requirements}
\label{sec:requirements}
%
The above systems tackle very different use cases.  As we discuss how they
partition their workloads and balance them across servers, however, it is easy
to see they have a number of common requirements, which we explicitly list here.

\squishlist
\item \textbf{Assignment of logical resources to physical servers}
Our use cases all involve taking a system's set of logical resources and mapping
them to physical servers.  The logical entities can be database partitions as in
Espresso, or a consumer as in the \databus consumption case.  Note a logical
entity may or may not have state associated with it, and a cluster manager must
be aware of any cost associated with this state (e.g. movement cost).
\item \textbf{Fault detection and resource reassignment}
All of our use case systems must handle cluster member failures by first
detecting such failures, and second re-replicating and reassigning resources 
across the surviving members, all while satisfying the system's invariants and 
load balancing goals.  For example, Espresso mandates a single master per
partition, while \databus consumption mandates a consumer must exist for every
database partition.  When a server fails, the masters or consumers on that
server must be reassigned. 
\item \textbf{Elasticity} 
Similar to failure detection and response requirement, systems must be able to
incorporate new cluster physical entities by redistributing logical resources
to those entities.  For example, Espresso moves partitions to new storage nodes,
and \databus moves database partitions to new consumers.  
\item \textbf{Monitoring}
Our use cases require we monitor systems to detect load imbalance, either
because of skewed load against a system's logical partitions (\eg an Espresso
hot spot), or because a physical server become degraded and cannot
handle its expected load (\eg via disk failure).  We must detect these
conditions, \eg by monitoring throughput or latency, and then invoke cluster
transitions to respond.
\squishend
 
Reflecting back on these requirements we observe a few key trends.  They all
involve encoding a system's optimal and minimally acceptable state, and having the ability
to respond to changes in the system to maintain the desired state.  In the
subsequent sections we show how we incorporate these requirements into \helix. 


