\documentclass[12pt]{article}
\begin{document}
\section{State Transitions --- Example}

The following example illustrates how Helix handles state transitions.
\begin{itemize}
\item Initial State: We have 1 partition in state Master on one machine. 
\item System Change: A second machine comes online 
\item Desired State: partition is Slave on first machine and
Master on second
\item 
Table~\ref{current_to_desired} illustrates the current and desired state
%-----------------------------------
\begin{table}[h]
\centering
\begin{tabular}{|l|l|l|l|l|l|} \hline \hline
\multicolumn{2}{|c|}{Current} & & \multicolumn{2}{|c|}{Desired} \\ \hline 
{\bf M1} & {\bf M2} & & {\bf M1} & {\bf M2} \\ \hline
Master   & Offline & \(\rightarrow\) & Slave & Master \\ \hline 
\hline
\end{tabular}
\label{current_to_desired}
\end{table}
%-----------------------------------
\item Table~\ref{state_transitions}  illustrates the state transitions.
\begin{enumerate}
\item Row 1 is all the transitions that need to be performed. However, because
of AFSM constraints, these cannot be fired in any order. 
\item Row 2 repreents
a legal grouping of the transitions that maximizes parallelism. 
\item Row 3 shows that maximal parallel execution is not always desirable. In this case,
we know that the Offline to Slave transition is the most expensive
and that it is more important to have a master than a slave. As
described earlier, Helix allows us to prioritize state transitions.
Prioritizing \((O \rightarrow S)\) above \(M \rightarrow  S\) yields a
transition sequence that respects the constraints and business
requirements of the DSS.
\end{enumerate}
%-----------------------------------
\begin{table}[h]
\centering
\begin{tabular}{|l|l||l|l|l|} \hline \hline
1 & To Execute & \multicolumn{3}{|c|}{\((1: M \rightarrow S), 
(2: O \rightarrow S), (2: S \rightarrow M)\)} \\ \hline 
2 & Legal & \multicolumn{2}{|c|}{\((1: M \rightarrow S), (2: O \rightarrow S)\)} & \((2: S \rightarrow M)\) \\ \hline
3 & Preferred & \((2: O \rightarrow S)\) & \((1: M \rightarrow S)\) & \((2: S \rightarrow M)\) \\ \hline
\hline
\end{tabular}
\label{state_transitions}
\end{table}
%-----------------------------------
\item {\bf Throttling}. Assume that we had another partition which we wished to undergo
the same transition. Since partitions are independent of each other, one
could execute the transitions for these two at the same time. However,
for the system as a whole, having 2 partitions go from Offline to Slave
at the same time might be too expensive. Helix allows us to restrict the
number of concurrent transitions of any type. This means that while it
is legal to execute \((2: O \rightarrow S)\) for both partitions
concurrently, Helix would not do this if the DDS dictated that it would
be inadvisable.

\end{itemize}


\end{document}
