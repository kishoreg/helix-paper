Distributed data systems systems are used in a variety of
settings like online serving, offline analytics, data 
transport, and search, among other use cases.  They let organizations scale out their 
workloads using cost-effective commodity hardware, while retaining key properties 
like fault tolerance and scalability.  At \linkedin we have built a number of
such systems.  A key pattern we observe is that even though they may serve different purposes, 
they tend to have a lot of common functionality, and tend to use common building blocks 
in their architectures.  One such building block that is just beginning to
receive attention is cluster management, which addresses 
the complexity of handling a dynamic, large-scale system with many servers.  
Such systems must handle software and hardware failures, 
setup tasks such as bootstrapping data, and
operational issues such as data placement, load balancing, planned upgrades, and cluster expansion.  
 
All of this shared complexity, which we see in all of our systems, motivates us
to build a cluster management framework, \helix, to solve these problems once in a
general way.

\helix provides an abstraction for a system developer to separate coordination and management tasks
from component functional tasks of a distributed system.  The developer defines the system behavior via a 
state model that enumerates the possible states of each component, the transitions between those states, 
and constraints that govern the system�s valid settings. \helix does the heavy lifting of
ensuring the system satisfies that state model in the distributed setting,
while also meeting the system's goals on load balancing and throttling state
changes.
We detail several \helix-managed production distributed systems 
at \linkedin and how \helix has helped them avoid building custom management components.  
We describe the \helix design and implementation and present an experimental
study that demonstrates its performance and functionality.

